\documentclass{article}

\usepackage{array}
\usepackage{etoolbox}
\usepackage{fancyhdr}
\usepackage{geometry} 
\usepackage{graphicx}
\usepackage{soul}
\usepackage{titling}
\usepackage{url}

%%%%%%%%%%%%%%%%%%%%%%%%%%%%%%%%%%%%%%%%%%%%%%%%%%%%%%%%%%%%
% BEGIN METADATA: Edit the following as appropriate
%%%%%%%%%%%%%%%%%%%%%%%%%%%%%%%%%%%%%%%%%%%%%%%%%%%%%%%%%%%%

\title{Project Title}  % the title of your project
\newcommand\shorttitle{\thetitle}  % if needed: a shorter title for the document header
% Team members.
\newcommand\firstname{Syed Ahad Ali}  % full name
\newcommand\firstid{sa07753}         % ID, e.g. xy01234
\newcommand\secondname{Asad Ullah Chaudhry} % full name
\newcommand\secondid{ac07408}        % ID, e.g. xy01234
\newcommand\thirdname{Muhammad Arsalan Hussain}  % full name
\newcommand\thirdid{mh07607}         % ID, e.g. xy01234
\newcommand\fourthname{Muzzammil Kamran Sattar}  % full name
\newcommand\fourthid{ms07164}         % ID, e.g. xy01234

%%%%%%%%%%%%%%%%%%%%%%%%%%%%%%%%%%%%%%%%%%%%%%%%%%%%%%%%%%%%
% END METADATA: Do not edit the preamble any further.
%%%%%%%%%%%%%%%%%%%%%%%%%%%%%%%%%%%%%%%%%%%%%%%%%%%%%%%%%%%%

\pagestyle{fancy}
\lhead{Kaavish Proposal}
\chead{\shorttitle}
\rhead{Fall 2024}
\cfoot{Page \thepage}
\renewcommand{\footrulewidth}{0.4pt}

\newcommand\instruction[1]{\textit{#1}}

\begin{document}

% Cover page.
\input{cover}

%%%%%%%%%%%%%%%%%%%%%%%%%%%%%%%%%%%%%%%%%%%%%%%%%%%%%%%%%%%%
% DATA: Populate the rest of the document as instructed.
%%%%%%%%%%%%%%%%%%%%%%%%%%%%%%%%%%%%%%%%%%%%%%%%%%%%%%%%%%%%
\section{Abstract}
\instruction{Please write a 500–600 word abstract on the project idea. It should not be very technically written but should be understandable by anyone.}

\section{Problem definition}

Pakistan has an employed population (formal and informal) of 71.76 million \cite{governmentofpakistan2021PakistanLabourForce2021} with a significant portion of the workforce
employed in the manufacturing (14.9\%) and construction (14.4\%) sectors \cite{governmentofpakistan2021PakistanLabourForce2021}. 
Out of the total workforce population, 1.9 (2.7\%) million people faced a work related injury in 2020-2021. Out of these 1.9 million people who were injured,
about 40.5\% of workers had to consult a medical professional, 11.7\% had to get hospitalized and 31.6\% were forced to take took time off work becuase of their injury
\cite{governmentofpakistan2021PakistanLabourForce2021}. The labour workforce in the industrial sector (engaged in assembly and manufacturing) is one of the most dangerous 
sectors to work in, accounting for 20\% of all workplace injuries in 2020-2021 \cite{governmentofpakistan2021PakistanLabourForce2021}.
With the monthly average income in the manufacturing sector being 22,000 PKR \cite{governmentofpakistan2021PakistanLabourForce2021}, 
such injuries can have a significant impact on the financial stability of the workers and their families with many not being able to afford healthcare services at all. 
Formal sector employers usually hestitate to provide insurance for their workers, and the government does not have a comprehensive social security system in place to support the injured workers.
The informal sector workers (the majority of the labour segment) are left to fend for themselves entirely. Informal sector jobs often don't involve formal work agreements, 
leading to the workers being deprived of their rights as set out by Pakistan's constitution entirely. For this sector, the government has no data on the number of injuries or deaths that occur
It should also be noted that Pakistan stands as one of the countries with the least amount of regulation and ;protection when it comes to labor rights. For a labor force of 10.63 million (that is part of the formal sector),
Pakistan has nearly 517 labour inspectors which means one inspector for more than 20,560 workers\cite{Whistlers} contributing to the neglect of Labour rights, safety protocols and education of the workers into safe practices.

This project aims to address the issue of workplace safety in the industrial sector by developing a real-time safety monitoring system that can detect unsafe behaviors and conditions in the workplace. 
Such a system can help prevent accidents and injuries by alerting workers and supervisors when unsafe conditions are detected, allowing them to take corrective action \textbf{before} an accident occurs.
Such a system should also be able to detect if an accident has occured and alert the relevant authorities so that immediate action can be taken to help the injured worker. For the manufacturers, such a system can help 
them comply with safety regulations and reduce workplace injuries. For the workers, such a system can help them stay safe and avoid accidents, reducing the financial burden of healthcare costs
and lost wages. 



We are monitoring: Hard hat, Safety Vest, Fall Detection/Accident Detection, Hazardous Zones Entry
In the context of the company we are working with, Dawlence assures us that all their workers are given safety equipment. Our project aims to make sure that the workers are using this equipment when on the factory floor. 

\section{Social relevance}


By detecting unsafe behaviors and conditions in real-time, this project aims to reduce workplace injuries in industrial enviroments, addressing a critical societal issue related to the safety and health of workers. 
Fewer workplace accidents also lead to reduced healthcare costs and less downtime due to injuries, lowering the economic burden on both companies and workers (who already have limited funds to spend on healthcare).

Given the small number of labour inspectors for the millions of workers in Pakistan, automating safety monitoring safety monitoring ensures better compliance with safety regulations,
reducing the need for a large fleet of human inspectors which suits the current socio-economic conditions of Pakistan's government. 
The proposed system will also provide insights and analytics, which could help Dawlence identify trends and continuously improve workplace safety. This approach encourages a shift towards more proactive safety management, benefiting both employees and employers. 

\section{Originality/Novelty}
\instruction{Describe the value of solving the problem. Compare and contrast with any existing solutions.}

\section{CS contribution}
\instruction{Describe the CS component of the project, e.g.\ the higher level CS courses that contribute to it.}
Our project is the integration of various CS and potentially CE components. 

\begin{enumerate}
    \item \textbf{Introduction to Deep Learning}: The heart of the project is fine-tuning AI models to make them capable of identifying safety hazards in real time through image and video data. The core concepts from deep learning, including Convolutional Neural Networks (CNNs), object detection, and transfer learning, will be used to create a highly accurate detection system. The project's main code will be leveraging frameworks such as TensorFlow, PyTorch and Ultralytics for fine-tuning and inference.
    \item \textbf{Computer Vision}: The AI models' functionality revolves around recognizing and classifying images, including detecting protective equipment (like hard hats and vests), state of machines (like ON or OFF) and identifying hazardous zones. The project will employ techniques such as image segmentation, object detection and classification, distance calculation etc. Furthermore, OpenCV, Ultralytics and other image processing libraries will be used to extract pre-process visual data captured by CCTV cameras and perhaps even for inference.
    \item \textbf{Data Communication and Networking}: Real-time data will need to be transmitted across different parts of the system, i.e. from cameras to servers and alerting systems. Thus, it is essential to establish a robust link between CCTV video footage and our backend server. This link will handle all the preprocessing (i.e. video to frames conversion) and send data to our backend for model inference whilst ensuring low latency transmission. Furthermore, our model may be hosted on the cloud via AWS or Google cloud services which would require further data transmission from our backend server. Understanding data communication protocols, cloud computing integration, and latency minimization will be crucial. Secure and efficient data transmission using networking protocols ensures that safety alerts can be sent out without delay and without corruption.
    \item \textbf{Embedded Systems}: The project integrates with CCTV camera systems, and part of the implementation may involve working with embedded systems to ensure efficient data capture. For example, rotating cameras may be required to zoom in and focus on a workplace hazard or incident (Note that the zoom will not be digital but via camera lens). Static cameras may need to follow a variable routine of zooming in on different machines; perhaps machines whose current state is ON. This component's requirement is inversely proportional to the number of cameras that Dawlance installs in their warehouses for our project.    
    \item \textbf{Data annotation and preparation}:  This includes cleaning video footage, annotating safety-relevant scenes (e.g., workers without protective gear) and safety irrelevant scenes. These tasks will ensure that the AI models have relevant high quality data to work with. This is the most straightforward part of the project but also the most laborious.
    \item \textbf{Web Development}: The project will involve developing a user-friendly web-based dashboard to display real-time data and analytics regarding workplace safety compliance. This web app will allow users to monitor safety alerts and trends.
    \item \textbf{Software Engineering}: The overall architecture of the project will follow best practices in software engineering, including modular design, agile development, continuous integration, and deployment (CI/CD). Due to the large scope of the project, it would be best to deliver modest features in multiple sprints starting with easier functionality like PPE detection.
\end{enumerate}

\section{Scope and Deliverables}
\instruction{Justify the scope of the project with respect to the size of the team and the year long duration. List the foreseeable deliverables.}
\begin{enumerate}
    \item \textbf{Dataset creation from scratch}: 
    \item \textbf{Integration of various technologies}:    
    \item \textbf{Interdiscplinary Nature}: The project is not limited to just software development; it also involves deep learning, computer vision, hardware setup (camera systems), and safety protocol knowledge. This interdisciplinary nature highlights the need for expertise in various domains, including AI, computer networks, and embedded systems.    
\end{enumerate}

\subsection{Foreseeable Deliverables}
\begin{enumerate}
    \item \textbf{S.R.S Document}:    
    \item \textbf{Deployed dashboard web application}:
    % this will contain a widget for data driven insights
    \item \textbf{Notification system}:
    % might include an alarm system, voice alert, web app notifications, email
    \item \textbf{The hazard detection AI model}:
    \item \textbf{Post processing algorithm}:
    % this will include logic to check if the requirements are met based on the objects detected from AI model. If not, required action will be taken i.e. email notification, voice alarm etc.
    \item \textbf{Documentation}:
\end{enumerate}

\section{Feasibility}
\instruction{List the resources, e.g.\ datasets, compute resources, software libraries, hardware, required for the project. Mention how you expect to access and utilize them for the project.}

\section{Team dynamics}
\instruction{Justify the suitability of the team members to the project. For example, their relevant courses, projects, internships, or research.}

\section{Tech stack}
\instruction{Write details of the tech stack you will use for this project for e.g.\ if you are using MERN stack, you can write MongoDB, Express, React and NodeJS etc.}

\section{References}
\instruction{List your references.}


% External advisor undertaking.
\input{external}

% Refrences
\newpage
\bibliographystyle{ieeetr}
\bibliography{sources.bib}

\end{document}

%%% Local Variables:
%%% mode: latex
%%% TeX-master: t
%%% End:
