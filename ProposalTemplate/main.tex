\documentclass{article}

\usepackage{array}
\usepackage{etoolbox}
\usepackage{fancyhdr}
\usepackage{geometry} 
\usepackage{graphicx}
\usepackage{soul}
\usepackage{titling}
\usepackage{url}

%%%%%%%%%%%%%%%%%%%%%%%%%%%%%%%%%%%%%%%%%%%%%%%%%%%%%%%%%%%%
% BEGIN METADATA: Edit the following as appropriate
%%%%%%%%%%%%%%%%%%%%%%%%%%%%%%%%%%%%%%%%%%%%%%%%%%%%%%%%%%%%

\title{Project Title}  % the title of your project
\newcommand\shorttitle{\thetitle}  % if needed: a shorter title for the document header
% Team members.
\newcommand\firstname{Syed Ahad Ali}  % full name
\newcommand\firstid{sa07753}         % ID, e.g. xy01234
\newcommand\secondname{Asad Ullah Chaudhry} % full name
\newcommand\secondid{ac07408}        % ID, e.g. xy01234
\newcommand\thirdname{Muhammad Arsalan Hussain}  % full name
\newcommand\thirdid{mh07607}         % ID, e.g. xy01234
\newcommand\fourthname{Muzzammil Kamran Sattar}  % full name
\newcommand\fourthid{ms07164}         % ID, e.g. xy01234
% Uncomment the rows for the next 2 students if and as needed.
% \newcommand\fourthname{Student 4} % full name
% \newcommand\fourthid{id04}        % ID, e.g. xy01234
% \newcommand\fifthname{Student 5}  % full name
% \newcommand\fifthid{id05}         % ID, e.g. xy01234

%%%%%%%%%%%%%%%%%%%%%%%%%%%%%%%%%%%%%%%%%%%%%%%%%%%%%%%%%%%%
% END METADATA: Do not edit the preamble any further.
%%%%%%%%%%%%%%%%%%%%%%%%%%%%%%%%%%%%%%%%%%%%%%%%%%%%%%%%%%%%

\pagestyle{fancy}
\lhead{Kaavish Proposal}
\chead{\shorttitle}
\rhead{Fall 2024}
\cfoot{Page \thepage}
\renewcommand{\footrulewidth}{0.4pt}

\newcommand\instruction[1]{\textit{#1}}

\begin{document}

% Cover page.
\begin{titlepage}

\center % Center everything on the page
 
%----------------------------------------------------------------------------------------
%	HEADING SECTIONS
%----------------------------------------------------------------------------------------

\textsc{
  {\LARGE \bf \thetitle}\\\bigskip\bigskip % Your Project Title
  {\large
    Kaavish Project Proposal\\\bigskip
    By}
}\\\bigskip 

%----------------------------------------------------------------------------------------
%	AUTHOR SECTION
%----------------------------------------------------------------------------------------

{\large
  \begin{tabular}{ll}
    \firstname & (\firstid@st.habib.edu.pk) \\
    \secondname & (\secondid@st.habib.edu.pk) \\
    \thirdname & (\thirdid@st.habib.edu.pk) \\
    \ifdef{\fourthname}{\fourthname & (\fourthid@st.habib.edu.pk) \\}{}
    \ifdef{\fifthname}{\fifthname & (\fifthid@st.habib.edu.pk) \\}{}
  \end{tabular}
}
\bigskip\bigskip\bigskip

{\large \today}\\\bigskip\bigskip

\includegraphics[height=5cm]{HU_logo}\\\bigskip
 
%----------------------------------------------------------------------------------------
{\large
  In partial fulfillment of the requirement for \\\medskip
Bachelor of Science \\\medskip
Computer Science
}\\\bigskip\bigskip\bigskip

{\large
  \textsc{
    Dhanani School of Science and Engineering\\\bigskip
    Habib University\\\bigskip 
    Fall 2024
  }\\\bigskip\bigskip 
  Copyright @ 2024 Habib University
}

\end{titlepage}


%%%%%%%%%%%%%%%%%%%%%%%%%%%%%%%%%%%%%%%%%%%%%%%%%%%%%%%%%%%%
% DATA: Populate the rest of the document as instructed.
%%%%%%%%%%%%%%%%%%%%%%%%%%%%%%%%%%%%%%%%%%%%%%%%%%%%%%%%%%%%
\section{Abstract}
\instruction{Please write a 500–600 word abstract on the project idea. It should not be very technically written but should be understandable by anyone.}

\section{Problem definition}
\instruction{Describe the problem that the project addresses.}




\section{Social relevance}
\instruction{Describe any societal problem that the project addresses.}
Pakistan stands as one fo the countires with the least amount of regulation and protection when it comes to labor rights and labor unions. 
WIth a abour force of 65.5 million workers of which only 10.63 million are engaged in the formal sector. For these 10.63 million formal sector workers, Pakistan has nearly 517 labour inspectors which translate to one inspector for more than 20,560 workers in the so-called formal sector \cite{noauthor_whistlers_nodate} 

\section{Originality/Novelty}
\instruction{Describe the value of solving the problem. Compare and contrast with any existing solutions.}

\section{CS contribution}
\instruction{Describe the CS component of the project, e.g.\ the higher level CS courses that contribute to it.}

\section{Scope and Deliverables}
\instruction{Justify the scope of the project with respect to the size of the team and the year long duration. List the foreseeable deliverables.}

\section{Feasibility}
\instruction{List the resources, e.g.\ datasets, compute resources, software libraries, hardware, required for the project. Mention how you expect to access and utilize them for the project.}

\section{Team dynamics}
\instruction{Justify the suitability of the team members to the project. For example, their relevant courses, projects, internships, or research.}

\section{Tech stack}
\instruction{Write details of the tech stack you will use for this project for e.g.\ if you are using MERN stack, you can write MongoDB, Express, React and NodeJS etc.}

\section{References}
\instruction{List your references.}

% External advisor undertaking.
\input{external}

% Refrences
\newpage
\bibliographystyle{ieeetr}
\bibliography{sources.bib}

\end{document}

%%% Local Variables:
%%% mode: latex
%%% TeX-master: t
%%% End:
