\documentclass{article}

\usepackage{array}
\usepackage{etoolbox}
\usepackage{fancyhdr}
\usepackage{geometry} 
\usepackage{graphicx}
\usepackage{soul}
\usepackage{titling}
\usepackage{url}

%%%%%%%%%%%%%%%%%%%%%%%%%%%%%%%%%%%%%%%%%%%%%%%%%%%%%%%%%%%%
% BEGIN METADATA: Edit the following as appropriate
%%%%%%%%%%%%%%%%%%%%%%%%%%%%%%%%%%%%%%%%%%%%%%%%%%%%%%%%%%%%

\title{Project Title}  % the title of your project
\newcommand\shorttitle{\thetitle}  % if needed: a shorter title for the document header
% Team members.
\newcommand\firstname{Syed Ahad Ali}  % full name
\newcommand\firstid{sa07753}         % ID, e.g. xy01234
\newcommand\secondname{Asad Ullah Chaudhry} % full name
\newcommand\secondid{ac07408}        % ID, e.g. xy01234
\newcommand\thirdname{Muhammad Arsalan Hussain}  % full name
\newcommand\thirdid{mh07607}         % ID, e.g. xy01234
\newcommand\fourthname{Muzzammil Kamran Sattar}  % full name
\newcommand\fourthid{ms07164}         % ID, e.g. xy01234

%%%%%%%%%%%%%%%%%%%%%%%%%%%%%%%%%%%%%%%%%%%%%%%%%%%%%%%%%%%%
% END METADATA: Do not edit the preamble any further.
%%%%%%%%%%%%%%%%%%%%%%%%%%%%%%%%%%%%%%%%%%%%%%%%%%%%%%%%%%%%

\pagestyle{fancy}
\lhead{Kaavish Proposal}
\chead{\shorttitle}
\rhead{Fall 2024}
\cfoot{Page \thepage}
\renewcommand{\footrulewidth}{0.4pt}

\newcommand\instruction[1]{\textit{#1}}

\begin{document}

% Cover page.
\begin{titlepage}

\center % Center everything on the page
 
%----------------------------------------------------------------------------------------
%	HEADING SECTIONS
%----------------------------------------------------------------------------------------

\textsc{
  {\LARGE \bf \thetitle}\\\bigskip\bigskip % Your Project Title
  {\large
    Kaavish Project Proposal\\\bigskip
    By}
}\\\bigskip 

%----------------------------------------------------------------------------------------
%	AUTHOR SECTION
%----------------------------------------------------------------------------------------

{\large
  \begin{tabular}{ll}
    \firstname & (\firstid@st.habib.edu.pk) \\
    \secondname & (\secondid@st.habib.edu.pk) \\
    \thirdname & (\thirdid@st.habib.edu.pk) \\
    \ifdef{\fourthname}{\fourthname & (\fourthid@st.habib.edu.pk) \\}{}
    \ifdef{\fifthname}{\fifthname & (\fifthid@st.habib.edu.pk) \\}{}
  \end{tabular}
}
\bigskip\bigskip\bigskip

{\large \today}\\\bigskip\bigskip

\includegraphics[height=5cm]{HU_logo}\\\bigskip
 
%----------------------------------------------------------------------------------------
{\large
  In partial fulfillment of the requirement for \\\medskip
Bachelor of Science \\\medskip
Computer Science
}\\\bigskip\bigskip\bigskip

{\large
  \textsc{
    Dhanani School of Science and Engineering\\\bigskip
    Habib University\\\bigskip 
    Fall 2024
  }\\\bigskip\bigskip 
  Copyright @ 2024 Habib University
}

\end{titlepage}


%%%%%%%%%%%%%%%%%%%%%%%%%%%%%%%%%%%%%%%%%%%%%%%%%%%%%%%%%%%%
% DATA: Populate the rest of the document as instructed.
%%%%%%%%%%%%%%%%%%%%%%%%%%%%%%%%%%%%%%%%%%%%%%%%%%%%%%%%%%%%
\section{Abstract}
\instruction{Please write a 500–600 word abstract on the project idea. It should not be very technically written but should be understandable by anyone.}

\section{Problem definition}

Pakistan has an employed population (formal and informal) of 71.76 million \cite{governmentofpakistan2021PakistanLabourForce2021} with a significant portion of the workforce
employed in the manufacturing (14.9\%) and construction (14.4\%) sectors \cite{governmentofpakistan2021PakistanLabourForce2021}. 
Out of the total workforce population, 1.9 (2.7\%) million people faced a work related injury in 2020-2021. Out of these 1.9 million people who were injured,
about 40.5\% of workers had to consult a medical professional, 11.7\% had to get hospitalized and 31.6\% were forced to take took time off work becuase of their injury
\cite{governmentofpakistan2021PakistanLabourForce2021}. The labour workforce in the industrial sector (engaged in assembly and manufacturing) is one of the most dangerous 
sectors to work in, accounting for 20\% of all workplace injuries in 2020-2021 \cite{governmentofpakistan2021PakistanLabourForce2021}.
With the monthly average income in the manufacturing sector being 22,000 PKR \cite{governmentofpakistan2021PakistanLabourForce2021}, 
such injuries can have a significant impact on the financial stability of the workers and their families with many not being able to afford healthcare services at all. 
Formal sector employers usually hestitate to provide insurance for their workers, and the government does not have a comprehensive social security system in place to support the injured workers.
The informal sector workers (the majority of the labour segment) are left to fend for themselves entirely. Informal sector jobs often don't involve formal work agreements, 
leading to the workers being deprived of their rights as set out by Pakistan's constitution entirely. For this sector, the government has no data on the number of injuries or deaths that occur
It should also be noted that Pakistan stands as one of the countries with the least amount of regulation and ;protection when it comes to labor rights. For a labor force of 10.63 million (that is part of the formal sector),
Pakistan has nearly 517 labour inspectors which means one inspector for more than 20,560 workers\cite{Whistlers} contributing to the neglect of Labour rights, safety protocols and education of the workers into safe practices.

This project aims to address the issue of workplace safety in the industrial sector by developing a real-time safety monitoring system that can detect unsafe behaviors and conditions in the workplace. 
Such a system can help prevent accidents and injuries by alerting workers and supervisors when unsafe conditions are detected, allowing them to take corrective action \textbf{before} an accident occurs.
Such a system should also be able to detect if an accident has occured and alert the relevant authorities so that immediate action can be taken to help the injured worker. For the manufacturers, such a system can help 
them comply with safety regulations and reduce workplace injuries. For the workers, such a system can help them stay safe and avoid accidents, reducing the financial burden of healthcare costs
and lost wages. 



We are monitoring: Hard hat, Safety Vest, Fall Detection/Accident Detection, Hazardous Zones Entry
In the context of the company we are working with, Dawlence assures us that all their workers are given safety equipment. Our project aims to make sure that the workers are using this equipment when on the factory floor. 

\section{Social relevance}


By detecting unsafe behaviors and conditions in real-time, this project aims to reduce workplace injuries in industrial enviroments, addressing a critical societal issue related to the safety and health of workers. 
Fewer workplace accidents also lead to reduced healthcare costs and less downtime due to injuries, lowering the economic burden on both companies and workers (who already have limited funds to spend on healthcare).

Given the small number of labour inspectors for the millions of workers in Pakistan, automating safety monitoring safety monitoring ensures better compliance with safety regulations,
reducing the need for a large fleet of human inspectors which suits the current socio-economic conditions of Pakistan's government. 
The proposed system will also provide insights and analytics, which could help Dawlence identify trends and continuously improve workplace safety. This approach encourages a shift towards more proactive safety management, benefiting both employees and employers. 

\section{Originality/Novelty}
\instruction{Describe the value of solving the problem. Compare and contrast with any existing solutions.}

\section{CS contribution}
\instruction{Describe the CS component of the project, e.g.\ the higher level CS courses that contribute to it.}
\begin{enumerate}
    \item \textbf{Introduction to Deep Learning}:
    \item \textbf{Computer Vision}: 
    \item \textbf{Data Communication and Networking}:
    \item \textbf{Embedded Systems}:
    % mention cloud services and information from machines
    \item \textbf{Data preparation}:
    % mention asad
    \item \textbf{Web Development}:
    \item \textbf{Software Engineering}:
\end{enumerate}

\section{Scope and Deliverables}
\instruction{Justify the scope of the project with respect to the size of the team and the year long duration. List the foreseeable deliverables.}
\begin{enumerate}
    \item \textbf{Interdiscplinary Nature}: 
    \item \textbf{Integration of various technologies}:
    \item 
\end{enumerate}

\subsection{Foreseeable Deliverables}
\begin{enumerate}
    \item 
\end{enumerate}

\section{Feasibility}
\instruction{List the resources, e.g.\ datasets, compute resources, software libraries, hardware, required for the project. Mention how you expect to access and utilize them for the project.}

Datasets:
    We will be using data from Dawlences CCTV cameras to test and train our models. Furthermore we will also use the 
    following Datasets:

    1.

    2.

    3.

Software Libraries:
    Deep Learning Models: We anticipate that this project will extensively require the use of Deep Learning models 
    including Yolov8. Yolov8 will be used for Object Detection for instance detecting the absence of safety equipment
    and enterance into restrcited zones: 
    
    PyTorch and Tensor Flow: We plan on using these libaries for 

    OpenCV:

Hardware:

\section{Team dynamics}
\instruction{Justify the suitability of the team members to the project. For example, their relevant courses, projects, internships, or research.}

Muhammad Arsalan Hussain possesses a broad knowledge base across multiple domains
within computer science, with a core focus on deep learning. His expertise includes extensive
work in dataset creation, augmentation, and fine-tuning large sequence-to-sequence models,
which he applied to achieve high accuracy in his Automated Urdu Grammar Correction project
for his Deep Learning Course. Beyond technical skills, Arsalan has consistently demonstrated
strong leadership capabilities in his projects and is currently the lead developer for an employee
management software at Nasfia Pvt Ltd. His experience in creating datasets, optimizing AI
models and leading development teams will be undoubtedly a great asset to your project.

Syed Ahad Ali specializes in training LLMs, machine learning models, and computer vision. He
has worked on several freelance projects which have required these skills such that he was able
to polish them quite well. Apart from that he has significant experience in developing 3D models
in Unity and have created web pages which use webGL library to display the models, giving him
insight into how to handle MERN stack.

Asad Ullah Chaudhry brings a robust background in Data Science, Artificial Neural Networks
and Reinforcement Learning. His familiarity with Deep Learning and Reinforcement Learning
would play a key role in this project, helping finetune and deploy the YOLOV8 model for
compliance monitoring. He has a robust amount of experience in Data Science and Network
Analysis and is currently in the process of publishing his own Research Paper. This experience
will be vital in creating a custom dataset and acquiring data-driven insights for the intenseye
project.

Muzzammil Sattar, has extensive experience in Software Engineering, Human-Computer
Interaction, and Computer Graphics. Their work in Computer Graphics includes developing
efficient algorithms and data structures for managing large datasets, which will be valuable in
optimizing real-time data processing for this project. Additionally, their knowledge in Software
Engineering will assist the team effectively manage requirements, changes, and
constraints. Their expertise in Human-Computer Interaction will also contribute to creating a
user-friendly, easy-to-understand system for managers at Dawlance to use. Lastly being well
versed in Database Systems, they will be able to support with Backend Development and Design. 




\section{Tech stack}
\instruction{Write details of the tech stack you will use for this project for e.g.\ if you are using MERN stack, you can write MongoDB, Express, React and NodeJS etc.}

\section{References}
\instruction{List your references.}


% External advisor undertaking.
\input{external}

% Refrences
\newpage
\bibliographystyle{ieeetr}
\bibliography{sources.bib}

\end{document}

%%% Local Variables:
%%% mode: latex
%%% TeX-master: t
%%% End:
